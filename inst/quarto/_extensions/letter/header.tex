% load packages
\usepackage{geometry}
\usepackage[pscoord]{eso-pic}
\usepackage{paracol}
\usepackage{setspace}
\usepackage{tabularx}
\usepackage{titlesec}
\usepackage[most]{tcolorbox}

% Just a command doing nothing
% It's to separate different levels of brackets
\newcommand{\nothing}[1]{}

% Define Swiss TPH colors
\definecolor{swisstphred}{HTML}{BF3227}
\definecolor{swisstphblue}{HTML}{468AB2}
\definecolor{pureblack}{HTML}{222222}
\definecolor{goldensand}{HTML}{EDCD64}
\definecolor{limegreen}{HTML}{B5C752}

% Define font sizes for headers
\titleformat{\section}
  {\normalfont\fontsize{11}{16.8}\bfseries\color{swisstphblue}}
  {\thesection.}
  {1em}
  {}

%% Set page size
\geometry{a4paper, total={170mm,257mm}, left=20mm, top=52.5mm, bottom=15mm, right=20mm}

%% A useful function that can flexibly put text box
%% Source: https://tex.stackexchange.com/questions/24663/how-to-place-a-floating-text-box-at-a-specified-location-in-page-coordinates
\newcommand{\placetextbox}[4]{% \placetextbox{<offset top>}{<offset left/right>}{<align>}{<stuff>}
  \setbox0=\hbox{#4}% Put <stuff> in a box
  \AddToShipoutPictureFG*{% Add <stuff> to current page foreground
    \if#3r
    \put(\LenToUnit{\paperwidth-#1},\LenToUnit{\paperheight-#2}){\vtop{{\null}\makebox[0pt][r]{\begin{tabular}{r}#4\end{tabular}}}}%
    \else
    \put(\LenToUnit{#1},\LenToUnit{\paperheight-#2}){\vtop{{\null}\makebox[0pt][l]{\begin{tabular}{l}#4\end{tabular}}}}%
    \fi
  }%
}%


%%
\AddToShipoutPicture{
     % logo
    \AtPageLowerLeft{
      \put(\LenToUnit{\dimexpr\paperwidth-0.93\paperwidth}, \LenToUnit{0.88\paperheight}){
        \includegraphics[width=0.25\paperwidth]{_extensions/letter/TPHlogo.png}
      }%
    }%
}
